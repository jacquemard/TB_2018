\chapter*{Cahier des charges}

\section*{Contexte}
Avec les nouvelles techniques d'apprentissage automatique (\textit{machine learning}) telles que le \textit{deep learning}, il est devenu faisable de réaliser des applications d'analyse d'images au travers de l'entraînement à l'aide d'une collection d'images déjà catégorisées.
Le but de ce projet est de d'utiliser ces méthodes afin de détecter des voitures sur un parking ouvert et ainsi d'en déduire l'occupation du parking.
Le système à développer sera installé dans une première phase sur le toit du bâtiment de la HEIG-VD (route de Cheseaux) afin de mesurer l'occupation des différents parkings.

\section*{Cahier des charges}

\begin{itemize}
    \item Evaluation et choix d'un modèle de caméra
    \item Développement d'une infrastructure de captures d'images (caméras, réseau, serveurs cloud, stockage des images)
    \item Etablissement d'un corpus d'images prétraitées pour l'entraînement des algorithmes
    \item Prise en main des techniques d'apprentissage automatique
    \item Conception et développement de l'algorithme d'apprentissage automatique
    \item Tests et validation.
\end{itemize}