\chapter{Résumé}

Pour les utilisateurs de parkings, il est souvent souhaité connaître le nombre de places libres disponibles, tout du moins s'il en reste. Aussi, être capable d'obtenir des statistiques d'utilisations des parkings peut être important pour leur gérant: par exemple, savoir distinguer les heures creuses des heures pleines permet d'optimiser les plages horaires des employés. C'est une problématique importante, qui a souvent été résolue par l'utilisation de divers capteurs. 

Ce travail propose une solution se reposant sur un système de capture vidéo. Des images provenant d'une caméra, il est possible d'en retirer le taux d'occupation du parking à l'aide de diverses méthodes d'analyse d'images ou d'apprentissage automatique. Ce projet permet d'exposer avant tout des méthodes de \textit{deep learning}, tel que la détection d'objets. 

Afin de concrétiser les solutions explorées dans ce projet, une caméra a été  installée sur le toit du site de Cheseaux la HEIG-VD. Celle-ci permet de capturer des images d'un des parkings qui y sont présents. L'évaluation des différentes caméras adéquates et l'installation de celle qui semble le plus adaptée est documentée dans ce travail. 

Ce projet propose d'étudier plusieurs solutions qui peuvent répondre à la problématique de la mesure du taux d'occupation d'un parking. L'une de celle-ci se repose sur une suppression de l'arrière-plan d'une image. Cependant, il est souhaité que ce travail aborde avant tout le problème à l'aide de technologies d'apprentissages automatiques, et non pas de traitement d'images. Néanmoins, la solution pré-citée est brièvement explorée.

Des modèles de réseaux de neurones sont aussi présentés. Certains de ceux-ci sont définies à la main. D'autres utilisent ce qui a été pensé dans le domaine de la recherche de la détection d'objets, dans le but de pouvoir distinguer, dans une images, les voitures qui y sont présentes. Notamment, les modèles \textit{Faster-RCNN} et \textit{Yolo} sont présentés et testés.
