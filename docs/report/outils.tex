\chapter*{Outils utilisés}
\addcontentsline{toc}{chapter}{Outils utilisés}  

Dans le cadre de ce travail de Bachelor, plusieurs outils ont été utilisés à l'aide au développement et à la rédaction. On les trouveras dans ce chapitre.

\paragraph{Visual Studio Code}
\textit{Visual Studio Code}\footnote{\textit{Visual Studio Code}: https://code.visualstudio.com/} est un éditeur Open Source comparable à \textit{Sublime Text} ou \textit{Atom}. 

Il est possible d'y ajouter des extensions afin de convenir à l'usage souhaité. Dans ce projet, cet éditeur a été utilisé avant tout afin de coder en \textit{Python} et rédiger en \textit{LaTex}. Les extensions suivantes ont donc été installées:
\begin{description}
    \item[Python] Permet d'intégrer à VSCode le développement en Python. Propose un système de \textit{Linting}, \textit{debugging}, ou encore d'auto-complétion.
    \item[autoDocstring] Permet de générer automatiquement de la documentation python (\textit{Python Docstrings})
    \item[LaTex Workshop] Permet de compiler un projet \textit{LaTex}, proposant de la coloration de code, ou encore un système de \textit{preview} automatique.
    \item[BibTexLanguage] Permet d'ajouter au fichier de bibliographie \textit{.bib} une coloration syntaxique.
    \item[Spell Right] Permet la correction orthographique multi-langues de fichiers, dont les documents \textit{LaTex}.
\end{description}

\paragraph{VIM}
\textit{VIM} est un éditeur de texte en ligne de commande. Il a avant tout été utilisé afin d'effectuer des modifications mineurs de code sur la VM.

\paragraph{Conda}
Ce gestionnaire d'environnement a été choisi afin de pouvoir interpréter du \textit{Python} facilement. La commande \textit{pip install} a pu être utilisée afin d'installer aisément des librairies supplémentaires. Il faut noter que cette commande est plus du ressort de \textit{Python} que de \textit{Conda}

\paragraph{Git}
\textit{Git} est un gestionnaire de version distribué Open Source. Grâce à celui-ci, il a été possible de suivre l'évolution de la réalisation de ce TB. Un \textit{repository} associé à ce travail à été déployé sur \textit{Github}.

\paragraph{LaTex}
Le site \url{http://www.tablesgenerator.com/latex_tables} a été utilisé afin de pouvoir générer facilement des tables \textit{LaTex}.


