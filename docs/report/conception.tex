\chapter{Conception}

\section{Architecture du système}\label{conception.architecture}

\subsection{Architecture logique}

\todo{VM, etc.}

\subsection{Architecture physique}
\subsubsection{Connexion de la caméra}\label{conception.architecture.physique.camera}
\todo{Schéma des différentes connexions possible + choix}

\subsubsection{Réseau}
\todo{Schéma réseau physique}


\todo{Architecture choisie : sur le toit, construction d'une salle en cours --> panneau solaire}

\section{Choix technologiques}

\subsection{Caméra réseau}
Une caméra réseau fournissant des images de qualités, et ce dans un prix raisonable, est nécessaire dans ce projet afin de capturer des images. 

\todo{Orientation vers des caméras de sécurités qui semblent le plus adaptées}

\subsubsection{Contraintes}
Installer une caméra sur le toit de la HEIG-VD a pour conséquence que certaines contraintes, tant d'ordre techniques qu'organisationnelles, doivent être respectées. Celles-ci sont décrites ci-dessous.

\paragraph{Capture de photo}
La caméra devra au minimum permettre la capture de photos d'une façon ou d'une autre. La capture de vidéos n'est pas nécessaire.  On notera que si celle-ci ne fournit malheureusement que la fonctionnalité de capture vidéo, il serait tout de même possible d'en extraire des images utilisables. 

\paragraph{Usage extérieur}
La caméra doit être prévue pour un usage extérieur, avec une protection contre les intempéries (pluie, neige, froid, chaud, etc.).

\paragraph{Connexion réseau}
En lien avec le paragraphe précédent, la caméra doit nécessairement avoir une interface Wifi permettant de se connecter au réseau de l'école. Une connexion par Ethernet n'est pas envisageable.

\paragraph{Caméra auto-alimentée}
Comme vu en section \ref{conception.architecture.physique.camera}, la caméra doit être auto-alimentée. Ainsi, une combinaison de batteries et de panneaux solaires semble être idéal. On peut noter que l'utilisation de batteries uniquement est envisageable dans le cadre d'un prototypage, mais ce uniquement si le remplacement de celles-ci doit être effectué à intervalles relativement éloignés, de l'ordre de la semaine.

\subsubsection{Critères}
Afin de choisir une caméra satisfaisante, plusieurs critères ont été définis. On les trouvera ci-après.

\paragraph{Qualité d'image}
La caméra doit avoir une bonne qualité d'image. On jugera:
\begin{itemize}
    \item La résolutions de l'image. Celle-ci doit être d'au minimum 1280x720 pixel.
    \item La qualité de l'image (si possible).
\end{itemize}

\paragraph{Fonctionnalités réseau}
La caméra doit pouvoir fournir des images via le réseau et il doit être possible d'en récupérer à intervalles réguliers. Pour ce faire, on pensera à des protocoles comme les suivants: 
\begin{description}
    \item[ONVIF (Open Network Video Interface Forum)] Standard industriel ouvert permettant de contrôler, configurer et communiquer avec des caméras de sécurité IP. Permet notamment la lecture de flux vidéo en temps réel et la capture de photos \footnote{\cite{wiki:onvif}, \citetitle{wiki:onvif}}
    \item[RTSP (Real Time Streaming Protocol)] Développé par RealNetworks, Netscape et Clumbia UNiversity, protocole de communication permettant de lire en temps réel un flux vidéo. \footnote{\cite{wiki:RTSP}, \citetitle{wiki:RTSP}}
    \item[requêtes HTTP] Il pourrait être possible de capturer et de récupérer à la demande une photo à l'aide de requêtes HTTP.
    \item[flux sur HTTP] Permet la lecture de flux vidéo sur HTTP en s'appuyant sur des formats tel que MPEG-4.
    \item[interface Web HTTP] Permet la configuration d'une caméra IP via un serveur Web qu'elle expose. 
    \item[FTP] La caméra peut exposer un serveur FTP contenant les photos capturées. Elle pourrait aussi "pusher" des photos capturées sur un serveur FTP distant.
\end{description}

\paragraph{Fonctionnement auto-alimenté}

\paragraph{Capture de photo}
\todo{capture de photo automatique}
\paragraph{Angle de vue}
\paragraph{Vision de nuit}
\paragraph{Facilité d'installation}

\paragraph{Prix}
