\chapter{Tests et validation}
Ce chapitre permet d'évaluer et comparer les modèles de détections d'objets \textit{faster RCNN} et \textit{Yolo v3}, tels que décrits en section \ref{conception.model.object} \itnameref{conception.model.object}. Il est important de noter qu'il a été remarqué qu'en général, la détection d'objets repose sur un entrainement sur des images non-prétraitées (pas de détection de bords). C'est pourquoi il semble plus aisé d'effectuer les tests sur le \textit{dataset} \textit{PKLot}.

Ont été évalués:
\begin{itemize}
    \item Le modèle pré-entrainé.
    \item Le modèle ré-entrainé sur le corpus d'image \textit{PKLot} avec différents nombres d'itérations. Le nombre d'itérations nécessaires diffère en fonction des modèles utilisés.
\end{itemize}

Chacun des modèles pré-cités ont été évalués à l'aide de la méthode décrite en section \ref{conception.eval} \itnameref{conception.eval}.

\section{TensorFlow Object Detection API}
Le modèle \textit{Faster RCNN} a été évalué à l'aide de l'API de détection d'objets de \textit{TensorFlow}. On trouvera en figure \ref{fig:tensorflow_train} l'entrainement qui a été effectué. 

\begin{figure}[H]
    \includegraphics[width=15cm]{img/tests/tensorflow_pklot_full_train.png}
    \centering
    \caption{Entrainement sur le corpus \textit{PKLot} (graphes par \textit{Tensorboard})}
    \label{fig:tensorflow_train}
\end{figure} 

On distingue plusieurs fonctions objectifs:
\begin{description}
    \item[\textit{Classification loss}] La perte au fil des itérations lors de la classification. 
    \item[\textit{Localization loss}] Présente à quel point les \textit{bounding box} désignant les voitures sont bien localisées.
    \item[\textit{Objectness loss}] Pour chaque \textit{bounding box}, il s'agit de définir si c'est une voiture ou l'arrière-plan
    \item[\textit{Total loss}] Présente une combinaison des toutes les fonctions de pertes.
\end{description}

L'entrainement a été stoppé après 16000 itérations (plus de 2 jours de calculs). En effet, il semble qu'un certain palier ai été atteint. Aussi, plus d'itérations pourraient entrainer à un sur-apprentissage. Les nombres d'itérations 4000 et 16000 ont été évalués, ainsi que le modèle pré-entrainé.

Afin d'effectuer les tests, il est nécessaire de choisir un score minimum. Celui-ci indique, pour chaque objet détecté, à quel point l'algorithme est confiant sur sa prédiction. La figure \ref{fig:tensorflow_bb} présente une détection effectuée sur une image du \textit{dataset} \textit{PKLot}. 

\begin{figure}[ht]
    \includegraphics[width=10cm]{img/tests/tensorflow_bb.jpg}
    \centering
    \caption{Exemple de détection sur un entrainement de 4000 itérations}
    \label{fig:tensorflow_bb}
\end{figure} 

Les scores sont généralement supérieurs à 75\%. Il est possible de remarquer un faux positif d'un score de 44\%. Un seuil de 70\% a été défini, afin d'obtenir une certaine marge, et réduire le nombre de faux positif. Cependant, il faut remarquer que si le modèle \textit{Faster RCNN} est utilisé, ce seuil à du être abaissé à 20\% pour obtenir des résultats raisonnables. Ceci permet aussi d'indiquer que l'utilisation d'un modèle non ré-entrainé ne permet pas une grande confiance en l'algorithme.

\paragraph{Résultats}

Le tableau \ref{tab:rcnn_results} présente les résultats du modèles \textit{Faster RCNN} pré-entrainé (0 itérations), ainsi qu'entrainé avec 4000 et 1600 itérations supplémentaires sur le corpus d'images \textit{PKLot}. Les calculs qui ont été effectués ont été décrit en section \ref{conception.eval} \itnameref{conception.eval}.

\begin{table}[ht]
\centering
\begin{tabular}{@{}ll@{}}
\toprule
N° itération & RMSE \\ \midrule
$0$            & $32.06564071$ \\
$4000$         & $10.4427355$ \\
$16000$        & $22.88589282$ \\ \bottomrule
\end{tabular}
\caption{\textit{Faster RCNN} - Résultats}
\label{tab:rcnn_results}
\end{table}

Le modèle entrainé 4000 fois semble le meilleur lors de l'évaluation. La cause est sans doute un sur-entrainement du modèle à 16000 itérations, ce qui ne permet pas de le généraliser suffisamment.

\section{Yolo v3}

\section{Evaluation du modèle}



\subsection{title}

\section{Tests en conditions réels}
parler vite fait des résultats sur la caméra