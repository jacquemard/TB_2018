\chapter{Conclusions}
\section{Bilan du projet}
\subsection{Réalisation}



Aussi des points positifs n'est ce pas ? :P

Au final, les images prises par la caméra n'ont pas été utilisée.
C'est de ses erreurs qu'on apprend

Choix de la caméra pénible
Problème lors de l'installation
Temps disponible

La caméra enfaite elle est pas mal

Temps d'entrainement long, donc il faut bien plus de temps pour tout comparer

Planification que peu possible (beaucoup de recherche, de tests, ce qui est fait dépend vraiment des recherches et des tests) Du coup, on a d'abord cherché à avoir un résulat, n'importe le quel. C'était très long, premier résultats à environ ~2.5 semaine avant la fin.

Mise en place difficile, python parfois bof avec les imports, magouille



\subsection{Résultats obtenus}

Résultats bon, mais peut être pas assez précis pour une vraie application pro. En effet, il faut pas qu'on dise qu'il reste une place sur un parking alors qu'il est plein. Par contre, c'est assez bien pour faire des statistiques d'utilisations. C'est ce que je conseille, plutôt pour des stats.

Parler pour l'hivers, le temps qui change, aps éété testé pour l'hiver, aussi de nuit

\subsection{Améliorations possibles}

Meilleures évalutations. 

Statistique, interface web
\section{Bilan personnel}


J'ai eu 3 lib à apprendre
